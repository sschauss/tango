\documentclass[12pt]{article}
\usepackage[utf8]{inputenc}
\usepackage{hyperref}
\usepackage{amsmath}
\usepackage{listings}

\setlength\parindent{0pt}

\title{
	\Huge{Introduction to Web Science} \\
	\vspace{1em}
	\LARGE{Assignment 1} \\
	\vspace{1em}
	\Large{TANGO}
}

\author {
	Mariya Chkalova \\{\normalsize\href{mailto:mchkalova@uni-koblenz.de}{mchkalova@uni-koblenz.de}} \and
	Arsenii Smyrnov \\{\normalsize\href{mailto:smyrnov@uni-koblenz.de}{smyrnov@uni-koblenz.de}} \and
	Simon Schau\ss \\{\normalsize\href{mailto:sschauss@uni-koblenz.de}{sschauss@uni-koblenz.de}}
}

\date{}

\begin{document}

\maketitle
\pagenumbering{gobble}
\newpage

\pagenumbering{arabic}

\section{Ethernet Frame}

\begin{enumerate}
	\item Source MAC Address: \texttt{00:13:10:e8:dd:52}
	\item Destination MAC Address: \texttt{00:27:10:21:fa:48}
	\item Protocol: Address Resolution Protocol 
	\item The penultimate field is the targets MAC Address and the last field is the targets IP Address.
\end{enumerate}

\section{Cable Issue}

Let $c$ be the speed of light, $l$ the length of the cable and $t$ the time it takes for the first bit to travel the length $l$. 
As the length of the cables are equal and the networks bandwidth doesn't change the propagation delay, the calculation for both networks are the same.  
Given the speed of light $c = 3 \cdot 10^8 \frac{m}{s}$ and the formula for the propagation delay $t = \frac{l}{c}$, the propagation delay is $t = \frac{20}{3 \cdot 10^8}s \approx 67ns$

\section{Basic Network Tools}

\lstset{breaklines=true, frame=single}

\textbf{1.} The \% packet loss if at all it happened after sending 100 packets.

Home: 0\%

University:

\begin{lstlisting}[caption=ping home]
	ping -c 100 -i 0.2  www.wikipedia.de
	...
	100 packets transmitted, 100 received, 0% packet loss, time 19883ms
	rtt min/avg/max/mdev = 18.037/21.074/29.851/1.646 ms
\end{lstlisting}

\textbf{2.} Size of the packet sent to Wikipedia server.

Home: 64 bytes

University: 64 bytes

\begin{lstlisting}[caption=man ping]
	       -s packetsize
	                     Specifies the number of data bytes to be sent.  The default is 56, which translates into 64 ICMP data
			     bytes when combined with the 8 bytes of ICMP header data.
\end{lstlisting}

\textbf{3.} IP address of your machine and the Wikipedia server

Home: 192.168.2.115, 91.198.174.192

University:

\begin{lstlisting}[caption=ifconfig home]
	ifconfig
	...
	bond0: flags=5187<UP,BROADCAST,RUNNING,MASTER,MULTICAST>  mtu 1500
	inet 192.168.2.115  netmask 255.255.255.0  broadcast 192.168.2.255
	inet6 fd21:22dd:f528:1:d6b5:5652:241e:f450  prefixlen 64  scopeid 0x0<global>
	inet6 fd21:22dd:f528:1:f2de:f1ff:fe03:c9c9  prefixlen 64  scopeid 0x0<global>
	inet6 2003:c5:5bd7:2653:d8fd:5b7d:730d:9337  prefixlen 64  scopeid 0x0<global>
	inet6 2003:c5:5bd7:2653:f2de:f1ff:fe03:c9c9  prefixlen 64  scopeid 0x0<global>
	inet6 fe80::f2de:f1ff:fe03:c9c9  prefixlen 64  scopeid 0x20<link>
	ether f0:de:f1:03:c9:c9  txqueuelen 1000  (Ethernet)
	RX packets 7563  bytes 6410345 (6.1 MiB)
	RX errors 0  dropped 0  overruns 0  frame 0
	TX packets 5621  bytes 1106251 (1.0 MiB)
	TX errors 0  dropped 0 overruns 0  carrier 0  collisions 0
\end{lstlisting}

\begin{lstlisting}[caption=arp wikipedia.org home]
	arp wikipedia.org
	...
	wikipedia.org (91.198.174.192) -- no entry	
\end{lstlisting}

\textbf{4.} Query Time for DNS query of the above url.

Home: 3msec

University: 

\begin{lstlisting}[caption=dig home]
	dig wikipedia.org
	...
	; <<>> DiG 9.11.0 <<>> wikipedia.org
	;; global options: +cmd
	;; Got answer:
	;; ->>HEADER<<- opcode: QUERY, status: NOERROR, id: 7395
	;; flags: qr rd ra ad; QUERY: 1, ANSWER: 1, AUTHORITY: 0, ADDITIONAL: 0

	;; QUESTION SECTION:
	;wikipedia.org.			IN	A

	;; ANSWER SECTION:
	wikipedia.org.		38	IN	A	91.198.174.192

	;; Query time: 3 msec
	;; SERVER: 192.168.2.1#53(192.168.2.1)
	;; WHEN: Wed Nov 02 06:22:14 UTC 2016
	;; MSG SIZE  rcvd: 47
\end{lstlisting}

\textbf{5.} Number of Hops in between your machine and the server

Home: didn't finish after 100+ hops

University: 

\textbf{6.} MAC address of the device that is acting as your network gateway.

Home:

University: 

\begin{lstlisting}[caption=arp home]
	arp -n
	...
	Address                  HWtype  HWaddress           Flags Mask            Iface
	192.168.2.1              ether   d4:21:22:dd:f5:28   C                     bond0
\end{lstlisting}


\section{Simple Python Programming}

\lstset{language=python, breaklines=true, frame=single}
\lstinputlisting{tango_assignment1_4.py}

\end{document}
